
\documentclass[runningheads]{llncs}
\usepackage[T1]{fontenc}
\usepackage{graphicx}
\usepackage{hyperref}
\usepackage{amsmath}
\usepackage{amssymb}
\usepackage{mathtools}
\usepackage{booktabs}
\usepackage{graphicx}

\usepackage{color}
\renewcommand\UrlFont{\color{blue}\rmfamily}
\urlstyle{rm}

\newcommand{\lit}[1]{\textit{#1}}
\newcommand{\trm}[1]{\ensuremath{\textbf{#1}}}
\newcommand{\state}[3]{(#1, #2, #3)} % State tuple (e, p, k)
\newcommand{\context}[2]{(#1, #2)} % Context tuple (loc, env_id)
\newcommand{\lookupenv}{\text{lookup\_env}}
\newcommand{\lookupval}{\text{lookup\_val}}
\newcommand{\evalexpr}{\text{eval}}
\newcommand{\updateenv}{\text{update\_env}}
\newcommand{\createnewenv}{\text{create\_new\_env}}
\newcommand{\lookupctf}{\text{ctf}}
\newcommand{\mkwtext}[1]{\text{\small #1}}

\usepackage{makecell}
\renewcommand\theadalign{bc}
\renewcommand\theadfont{\normalsize\bfseries}

\begin{document}

\section*{Appendix B: Transition Relations}

Let the current state be $\state{e}{h}{k}$ where $k = \context{i}{\text{env\_id}} \cdot k_{\text{rest}}$. The next state $\state{e'}{h'}{k'}$ is determined by the instruction $P_i$ as follows:

\paragraph{Pass, Break, Continue, Global, Nonlocal}
\[ \state{e'}{h'}{k'} = \state{e}{h}{\context{\text{next}[i]}{\text{env\_id}} \cdot k_{\text{rest}}} \]

\paragraph{Expression Assignment}
\begin{displaymath}
	\begin{aligned}
		\state{e'}{h'}{k'}  & =
		\begin{cases*}
			\state{e_{\text{updated}}}{h}{\context{\text{next}[i]}{\text{env\_id}} \cdot k_{\text{rest}}} & \text{if } val $\neq$ error \\
			\state{e}{h}{\context{\text{err}[i]}{\text{env\_id}} \cdot k_{\text{rest}}}                   & \text{if } val = error
		\end{cases*}
		\\
		                    & \text{where}
		\\
		P_{i}               & = \lit{var} = \lit{expr}                                                       \\
		val                 & = \evalexpr(\lit{expr}, \text{env\_id}, e, h)                                  \\
		env_{\text{target}} & = \lookupenv(\lit{var}, \text{env\_id}, e, h)                                  \\
		e_{\text{updated}}  & = e[env_{\text{target}} \leftarrow env_{\text{target}}[\lit{var} \mapsto val]] \\
	\end{aligned}
\end{displaymath}

\paragraph{If, While}
\begin{displaymath}
	\begin{aligned}
		\state{e'}{h'}{k'} & =
		\begin{cases*}
			\state{e}{h}{\context{\text{true}[i]}{\text{env\_id}} \cdot k_{\text{rest}}}  & \text{if} val = true  \\
			\state{e}{h}{\context{\text{false}[i]}{\text{env\_id}} \cdot k_{\text{rest}}} & \text{if} val = false \\
			\state{e}{h}{\context{\text{err}[i]}{\text{env\_id}} \cdot k_{\text{rest}}}   & \text{otherwise}      \\
		\end{cases*}
		\\
		                   & \text{where}
		\\
		P_{i}              & = \text{if/while } \lit{expr}:                \\
		val                & = \evalexpr(\lit{expr}, \text{env\_id}, e, h) \\
	\end{aligned}
\end{displaymath}

\paragraph{Function Definition}
\begin{displaymath}
	\begin{aligned}
		\state{e'}{h'}{k'}  & =
		\begin{cases*}
			\state{e_{\text{updated}}}{h}{\context{\text{next}[i]}{\text{env\_id}} \cdot k_{\text{rest}}} & \text{if } val $\neq$ error \\
			\state{e}{h}{\context{\text{err}[i]}{\text{env\_id}} \cdot k_{\text{rest}}}                   & \text{if } val = error
		\end{cases*}
		\\
		                    & \text{where}
		\\
		P_{i}               & = \text{def } \lit{var} (id_{1}, id_{2}, \ldots, id_{n}):                          \\
		f_{\text{entry}}    & = \text{entry location of function block}                                          \\
		closure             & = Closure(f_{\text{entry}}, \text{env\_id}, [id_{1}, id_{2}, \ldots, id_{n}])      \\
		env_{\text{target}} & = \lookupenv(\lit{var}, \text{env\_id}, e, h)                                      \\
		e_{\text{updated}}  & = e[env_{\text{target}} \leftarrow env_{\text{target}}[\lit{var} \mapsto closure]] \\
	\end{aligned}
\end{displaymath}

\paragraph{Call Assignment}
\begin{displaymath}
	\begin{aligned}
		 & \state{e'}{h'}{k'}           =
		\begin{cases*}
			\state{e''}{h''}{k''}                                                       & \text{if } $|arg\_vals| = |formals| \land \forall v \in vals, v \neq \text{error}$ \\
			\state{e}{h}{\context{\text{err}[i]}{\text{env\_id}} \cdot k_{\text{rest}}} & otherwise
		\end{cases*}
		\\
		 & \text{where}
		\\
		 & P_{i}                          = \lit{var} = \lit{func\_var}(args)                               \\
		 & vals_{j}                       = \evalexpr(args_{j}, \text{env\_id}, e, h), 1 \leq j \leq |args| \\
		 & closure     = \lookupval(func\_var, \text{env\_id}, e, h)                                        \\
		 & Closure(f_{\text{loc}}, par\_env\_id, formals) = closure                                         \\
		 & \text{env\_new\_id}            = \createnewenv(e)                                                \\
		 & env_{\text{new}} = \text{populate\_env}(closure, \text{env\_new\_id})                            \\
		 & e'' = e + \{ \text{env\_new\_id} \mapsto env_{\text{new}} \}                                     \\
		 & h'' = h + \{\text{env\_new\_id} \mapsto \text{par\_env\_id}\}                                    \\
		 & k'' = \context{f_{\text{loc}}}{\text{env\_new\_id}} \cdot k
	\end{aligned}
\end{displaymath}

\paragraph{Return}
\begin{displaymath}
	\begin{aligned}
		 & \state{e'}{h'}{k'} =
		\begin{cases*}
			\state{e''}{h}{\context{\text{next}[\text{ret\_loc})]}{\text{ret\_env\_id}} \cdot k'_{\text{rest}}}
		\end{cases*}
		\\
		 & where                                                                                                           \\
		 & P_{i} = \text{return } expr                                                                                     \\
		 & val = \evalexpr(\lit{expr}, \text{env\_id}, e, h)                                                               \\
		 & \context{i}{\text{env\_id}} \cdot \context{\text{ret\_loc}}{\text{ret\_env\_id}} \cdot k'_{\text{rest}} = k     \\
		 & P_{\text{ret\_loc}} = assign\_var = \ldots                                                                      \\
		 & e'' = e[\lookupenv(\lit{assign\_var}, \text{ret\_env\_id}, e, h) \leftarrow env[\lit{assign\_var} \mapsto val]]
	\end{aligned}
\end{displaymath}

\section*{Descriptions of Helper Functions}

The formal transition rules in the previous section utilize several helper
functions to manage control flow, evaluate expressions, handle environments, and
perform lookups according to Python's scoping rules within the SimpliPy subset.
These functions are conceptually defined as follows:

\paragraph{\texttt{evalexpr}(\textit{expr}, \textit{env\_id}, \textit{e}, \textit{h})}
Evaluates a SimpliPy expression \textit{expr} within the current execution
context.
\begin{itemize}
	\item \textbf{Inputs:} The expression \textit{expr} to evaluate, the current
	      environment identifier \textit{env\_id}, the global lexical map \textit{e}, and
	      the lexical hierarchy \textit{h}.
	\item \textbf{Output:} The computed value of the expression, or a special
	      `error' marker if evaluation fails (e.g., type error, variable not found).
	\item \textbf{Behavior:}
	      \begin{itemize}
		      \item If \textit{expr} is a constant, returns the constant's value.
		      \item If \textit{expr} is a variable name, uses
		            \texttt{lookupval}(\textit{expr}, \textit{env\_id}, \textit{e}, \textit{h}) to
		            find its value. Returns `error' if \texttt{lookupval} fails.
		      \item If \textit{expr} involves operators, recursively calls
		            \texttt{evalexpr} on sub-expressions, performs the operation, and returns the
		            result. Propagates `error' if any sub-evaluation fails or if the operation is
		            invalid for the operand types.
	      \end{itemize}
\end{itemize}

\paragraph{\texttt{lookupenv}(\textit{var}, \textit{env\_id}, \textit{e}, \textit{h})}
Determines the target \textit{environment ID} where a variable \textit{var}
exists or where a new binding for it should be created during an assignment.
This function embodies Python's LEGB-like scope resolution for assignments and
lookups.
\begin{itemize}
	\item \textbf{Inputs:} The variable name \textit{var}, the current
	      environment identifier \textit{env\_id}, the lexical map \textit{e}, and the
	      lexical hierarchy \textit{h}.
	\item \textbf{Output:} The identifier of the environment where \textit{var}
	      is found or should be bound/updated, or `error'.
	\item \textbf{Behavior:}
	      \begin{itemize}
		      \item Consults static analysis results for the scope associated
		            with \textit{env\_id}.
		      \item If \textit{var} is declared \texttt{global} in this scope,
		            returns 0 (the global environment ID).
		      \item If \textit{var} is declared \texttt{nonlocal}, searches
		            ancestor environments starting from the parent of \textit{env\_id} (using
		            \textit{h}) upwards towards (but *not* including) the global scope. Returns the
		            ID of the first ancestor environment found that contains a binding for
		            \textit{var}. If not found, signals an error (e.g., returns -1 or raises an
		            exception).
		      \item Otherwise (neither \texttt{global} nor \texttt{nonlocal}):
		            \begin{itemize}
			            \item Traverses the lexical hierarchy starting from the
			                  current environment \textit{env\_id} upwards towards and including the global
			                  environment (ID 0), using the parent links in \textit{h}.
			            \item At each environment ID $curr\_env\_id$ in this
			                  traversal, check if \textit{var} exists as a key in the environment
			                  $e[curr\_env\_id]$.
			            \item If \textit{var} is found in $e[curr\_env\_id]$,
			                  return $curr\_env\_id$.
			            \item If the traversal completes (reaches global scope
			                  and checks it) without finding \textit{var}, return the *original* starting
			                  environment identifier \textit{env\_id}. This indicates that if an assignment
			                  occurs, a new binding should be created in the current local scope.
		            \end{itemize}
	      \end{itemize}
\end{itemize}

\paragraph{\texttt{lookupval}(\textit{var}, \textit{env\_id}, \textit{e}, \textit{h})}
Looks up the \textit{value} associated with a variable name \textit{var}. It
first determines the correct environment using \texttt{lookupenv} and then
retrieves the value.
\begin{itemize}
	\item \textbf{Inputs:} The variable name \textit{var}, the starting
	      environment identifier \textit{env\_id}, the lexical map \textit{e}, and the
	      lexical hierarchy \textit{h}.
	\item \textbf{Output:} The value bound to \textit{var}, or an `error' marker
	      if the variable is not found in any accessible scope (i.e., if
	      \texttt{lookupenv} indicates it doesn't exist).
	\item \textbf{Behavior:}
	      \begin{itemize}
		      \item Call \texttt{target\_env\_id = lookupenv(\textit{var},
			            \textit{env\_id}, \textit{e}, \textit{h})}.
		      \item If $target\_env\_id$ indicates that the variable was not found
		            during the lookup traversal, return `error'.
		      \item Otherwise (the variable was found in environment
		            $target\_env\_id$), retrieve the environment map $env = e[target\_env\_id]$.
		      \item Return the value associated with \textit{var} in that map:
		            $env[var]$.
	      \end{itemize}
\end{itemize}

\paragraph{\texttt{createnewenv}(\textit{e})}
Generates a unique identifier for a new environment frame.
\begin{itemize}
	\item \textbf{Input:} The current lexical map \textit{e}.
	\item \textbf{Output:} A new integer environment ID that is not currently a
	      key in \textit{e}.
	\item \textbf{Behavior:} Typically implemented by finding the maximum
	      existing ID in `dom(e)` and returning the next integer.
\end{itemize}

\paragraph{\texttt{populate\_env}(\textit{closure}, \textit{env\_new\_id}, \textit{arg\_vals})}
Initializes a new environment frame (\textit{env\_new\_id}) for a function call.
(Note: This function was used conceptually in the Call Assignment rule
description).
\begin{itemize}
	\item \textbf{Inputs:} The \textit{closure} being invoked (containing formal
	      parameters and definition environment ID), the ID for the new environment
	      \textit{env\_new\_id}, and the list of evaluated argument values
	      \textit{arg\_vals}.
	\item \textbf{Output:} A new environment dictionary (mapping variable names
	      to values) representing the initial state of the function's local scope.
	\item \textbf{Behavior:}
	      \begin{itemize}
		      \item Creates bindings in the new environment dictionary mapping
		            each formal parameter name (from the \textit{closure}) to the corresponding
		            value in \textit{arg\_vals}.
		      \item Identifies (via static analysis of the function's body) all
		            other variables defined locally within that function.
		      \item Initializes these other local variables in the new
		            environment dictionary to a special `uninitialized' marker ($\bot$).
	      \end{itemize}
\end{itemize}

\end{document}
